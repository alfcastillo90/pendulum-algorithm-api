\documentclass{ieeeaccess}
\usepackage{cite}
\usepackage{amsmath,amssymb,amsfonts}
\usepackage{algorithmic}
\usepackage{graphicx}
\usepackage{textcomp}
\def\BibTeX{{\rm B\kern-.05em{\sc i\kern-.025em b}\kern-.08em
    T\kern-.1667em\lower.7ex\hbox{E}\kern-.125emX}}
\begin{document}
\history{Fecha de publicación 27 de enero, 2023}
\doi{10.1109/ACCESS.2017.DOI}

\title{Desarrollo de una API basada en algoritmo del pendulo metaheurístico para la optimización de la distribución de medicamentos}
\author{\uppercase{Carlos Alfredo Castillo Rodriguez}}
\address[1]{Universidad Bernardo Ohiggins, Santiago de Chile, Chile, (e-mail: cacr1990@gmail.com)}
\tfootnote{}

\markboth
{Author \headeretal: Preparation of Papers for IEEE TRANSACTIONS and JOURNALS}
{Author \headeretal: Preparation of Papers for IEEE TRANSACTIONS and JOURNALS}

\corresp{Corresponding author: Carlos Alfredo Castillo Rodriguez (e-mail: cacr1990@gmail.com)}

\begin{abstract}
El objetivo de este trabajo es desarrollar una API Restful que implemente un algoritmo metaheurístico, el algoritmo de búsqueda del péndulo, con el fin de optimizar la distribución de medicamentos en centros de salud. Se analizará la situación actual en Chile, se evaluarán los requerimientos para implementar la aplicación, se realizará un análisis de factibilidad y se propondrá un modelo de aplicación web. Las preguntas de investigación incluyen los criterios actuales para distribuir medicamentos en los centros de salud de Chile, los requerimientos para implementar la solución propuesta, la factibilidad de desarrollo y la arquitectura de software de la solución. El método utilizado será el algoritmo de búsqueda del péndulo, inspirado en el balanceo de un péndulo, donde los pesos oscilan hacia adelante y atrás mientras que la amplitud de la oscilación disminuye con el tiempo hasta que alcanza el equilibrio.
\end{abstract}

\begin{keywords}
aprendizaje automático, aprendizaje profundo, redes neuronales, inteligencia artificial, visión computacional, procesamiento del lenguaje natural, minería de datos, big data, modelado predictivo, clasificación, agrupamiento, aprendizaje por refuerzo.
\end{keywords}

\titlepgskip=-15pt

\maketitle

\EOD

\end{document}
