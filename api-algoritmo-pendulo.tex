\documentclass{ieeeaccess}
\usepackage{cite}
\usepackage{amsmath,amssymb,amsfonts}
\usepackage{algorithmic}
\usepackage{graphicx}
\usepackage{textcomp}
\def\BibTeX{{\rm B\kern-.05em{\sc i\kern-.025em b}\kern-.08em
    T\kern-.1667em\lower.7ex\hbox{E}\kern-.125emX}}
\begin{document}
\history{Fecha de publicación 27 de enero, 2023}
\doi{10.1109/ACCESS.2017.DOI}

\title{Desarrollo de una API basada en algoritmo del pendulo metaheurístico para la optimización de la distribución de medicamentos}
\author{\uppercase{Carlos Alfredo Castillo Rodriguez}}
\address[1]{Universidad Bernardo Ohiggins, Santiago de Chile, Chile, (e-mail: cacr1990@gmail.com)}
\tfootnote{}

\markboth
{Author \headeretal: Preparation of Papers for IEEE TRANSACTIONS and JOURNALS}
{Author \headeretal: Preparation of Papers for IEEE TRANSACTIONS and JOURNALS}

\corresp{Corresponding author: Carlos Alfredo Castillo Rodriguez (e-mail: cacr1990@gmail.com)}

\begin{abstract}
El objetivo de este trabajo es desarrollar una API Restful que implemente un algoritmo metaheurístico, el algoritmo de búsqueda del péndulo, con el fin de optimizar la distribución de medicamentos en centros de salud. Se analizará la situación actual en Chile, se evaluarán los requerimientos para implementar la aplicación, se realizará un análisis de factibilidad y se propondrá un modelo de aplicación web. Las preguntas de investigación incluyen los criterios actuales para distribuir medicamentos en los centros de salud de Chile, los requerimientos para implementar la solución propuesta, la factibilidad de desarrollo y la arquitectura de software de la solución. El método utilizado será el algoritmo de búsqueda del péndulo, inspirado en el balanceo de un péndulo, donde los pesos oscilan hacia adelante y atrás mientras que la amplitud de la oscilación disminuye con el tiempo hasta que alcanza el equilibrio.
\end{abstract}

\begin{keywords}
aprendizaje automático, aprendizaje profundo, redes neuronales, inteligencia artificial, visión computacional, procesamiento del lenguaje natural, minería de datos, big data, modelado predictivo, clasificación, agrupamiento, aprendizaje por refuerzo.
\end{keywords}

\titlepgskip=-15pt

\maketitle

\section{Introduction}
\label{sec:introduction}
\PARstart{E}{l} problema presentado en este texto es el de la optimización, 
que se presenta en diversas áreas de estudio como la ingeniería, la ciencia de la salud, 
la planificación logística, la informática y las finanzas, entre otras. Estos problemas de optimización 
suelen implicar la maximización de ganancias o/y la minimización de pérdidas con ciertas restricciones. 
Muchos de estos problemas son complejos y resolverlos utilizando algoritmos de optimización exactos es impráctico debido al costo computacional
y el tiempo requerido.

La solución propuesta en este texto es el desarrollo de una solución de software que posea metaheurísticas en su nucleo, 
las metaheurísticas que son algoritmos de aproximación que proporcionan soluciones óptimas o cercanas a óptimas dentro de un tiempo y restricciones computacionales razonables. 
Las metaheurísticas son de propósito general y se pueden adaptar a diferentes problemas.

Se propone desarrollar esta propuesta utilizando un nuevo algoritmo metaheurístico llamado Algoritmo de Búsqueda de Péndulo (PSA, por sus siglas en inglés), 
que es un algoritmo basado en población en el que se busca la solución de un problema de optimización mediante un grupo de agentes que se mueven en el espacio de búsqueda 
buscando la solución óptima según el movimiento armónico de un péndulo. El algoritmo se ha validado utilizando 13 funciones de prueba multimodales 
y se ha comparado con PSO y SCA. Los resultados muestran que PSA es un buen algoritmo de optimización, 
superando a PSO en 8 de las 13 funciones y a SCA en 12 de las 13 funciones. 
Además, se ha aplicado PSA para la optimización de la distribución de vacunas en el caso de la epidemia de influenza H1N1 de Hong Kong en 2009 y se ha encontrado 
que la distribución encontrada por PSA es mejor para reducir el número de infecciones en comparación con tres estrategias tradicionales utilizadas en la ciencia de la salud.
\EOD

\end{document}
