\documentclass{ieeeaccess}
\usepackage{cite}
\usepackage{amsmath,amssymb,amsfonts}
\usepackage{algorithmic}
\usepackage{graphicx}
\usepackage{textcomp}
\def\BibTeX{{\rm B\kern-.05em{\sc i\kern-.025em b}\kern-.08em
    T\kern-.1667em\lower.7ex\hbox{E}\kern-.125emX}}
\begin{document}
\history{Fecha de publicación 27 de enero, 2023}
\doi{10.1109/ACCESS.2017.DOI}

\title{Desarrollo de una API basada en algoritmo del pendulo metaheurístico para la optimización de la distribución de medicamentos}
\author{\uppercase{Carlos Alfredo Castillo Rodriguez}}
\address[1]{Universidad Bernardo Ohiggins, Santiago de Chile, Chile, (e-mail: cacr1990@gmail.com)}
\tfootnote{}

\markboth
{Author \headeretal: Preparation of Papers for IEEE TRANSACTIONS and JOURNALS}
{Author \headeretal: Preparation of Papers for IEEE TRANSACTIONS and JOURNALS}

\corresp{Corresponding author: Carlos Alfredo Castillo Rodriguez (e-mail: cacr1990@gmail.com)}

\begin{abstract}
El objetivo de este trabajo es desarrollar una API Restful que implemente un algoritmo metaheurístico, el algoritmo de búsqueda del péndulo, con el fin de optimizar la distribución de medicamentos en centros de salud. Se analizará la situación actual en Chile, se evaluarán los requerimientos para implementar la aplicación, se realizará un análisis de factibilidad y se propondrá un modelo de aplicación web. Las preguntas de investigación incluyen los criterios actuales para distribuir medicamentos en los centros de salud de Chile, los requerimientos para implementar la solución propuesta, la factibilidad de desarrollo y la arquitectura de software de la solución. El método utilizado será el algoritmo de búsqueda del péndulo, inspirado en el balanceo de un péndulo, donde los pesos oscilan hacia adelante y atrás mientras que la amplitud de la oscilación disminuye con el tiempo hasta que alcanza el equilibrio.
\end{abstract}

\begin{keywords}
aprendizaje automático, aprendizaje profundo, redes neuronales, inteligencia artificial, visión computacional, procesamiento del lenguaje natural, minería de datos, big data, modelado predictivo, clasificación, agrupamiento, aprendizaje por refuerzo.
\end{keywords}

\titlepgskip=-15pt

\maketitle

\section{Introducción}
\label{sec:introduction}
\PARstart{E}{l}
problema de la optimización es un desafío recurrente en diversas áreas, como la ingeniería, la salud, la logística, la informática y las finanzas. Muchas veces, estos problemas son complejos y requieren una gran cantidad de tiempo y recursos para ser resueltos mediante algoritmos de optimización exactos.

Para abordar este desafío, se propone desarrollar una solución de software basada en metaheurísticas, que son algoritmos de aproximación capaces de proporcionar soluciones óptimas o cercanas a óptimas dentro de un tiempo y restricciones computacionales razonables. Estas técnicas son de propósito general y se pueden adaptar a diferentes problemas.

En particular, se plantea utilizar el Algoritmo de Búsqueda de Péndulo (PSA), una metaheurística basada en población que busca la solución de un problema de optimización mediante un grupo de agentes moviéndose en el espacio de búsqueda, siguiendo el movimiento armónico de un péndulo. Este algoritmo ha sido validado mediante pruebas con 13 funciones multimodales y ha demostrado ser superior a otros algoritmos como PSO y SCA en varias pruebas. Además, ha sido aplicado con éxito en la optimización de la distribución de vacunas durante una epidemia de influenza.

La idea es desarrollar una API Restful con NodeJS que implemente el algoritmo PSA para optimizar la distribución de medicamentos en centros de salud. Se analizará la situación actual en Chile, se evaluarán los requerimientos para implementar la aplicación, se realizará un análisis de factibilidad y se propondrá un modelo de aplicación web.

\section{Marco teórico}
\label{sec:TheoreticalFramework}
\PARstart{E}{n} primer lugar, se presentará una breve introducción a los problemas de optimización, incluyendo los diferentes tipos de problemas y las técnicas utilizadas para resolverlos. Se mencionarán los problemas de optimización exactos y los problemas de optimización aproximada, así como las ventajas y desventajas de cada uno.

En segundo lugar, se describirán las metaheurísticas y su importancia en la resolución de problemas de optimización. Se explicará cómo las metaheurísticas proporcionan soluciones óptimas o cercanas a óptimas dentro de un tiempo y restricciones computacionales razonables. Además, se mencionarán algunos de los algoritmos metaheurísticos más comunes, como el algoritmo genético, el algoritmo de colonia de hormigas y el algoritmo de búsqueda por enjambre.

En tercer lugar, se presentará una descripción detallada del algoritmo de búsqueda del péndulo, incluyendo su funcionamiento, las ventajas y desventajas de su uso, y los resultados obtenidos en estudios anteriores. Se comparará el algoritmo de búsqueda del péndulo con otros algoritmos metaheurísticos como PSO y SCA.

Finalmente, se discutirá la aplicabilidad del algoritmo de búsqueda del péndulo en la optimización de la distribución de medicamentos en centros de salud. Se analizará la situación actual en Chile, se evaluarán los requerimientos para implementar la aplicación, y se propondrá un modelo de aplicación web.

\section{Planteamiento del problema}
\label{sec:ProblemStatement}
\PARstart{E}{l} marco teórico propuesto en este trabajo tiene como objetivo abordar el problema del desabastecimiento de medicamentos esenciales en sistemas de salud. La OMS define el desabastecimiento como una insuficiencia en el suministro de medicamentos necesarios para satisfacer las necesidades de salud pública y de los pacientes. Este problema puede ser causado por diversas razones, pero en este trabajo se enfocará en una de ellas: la optimización de recursos.

Para abordar este problema de optimización, se propone utilizar una técnica de metaheurística. La metaheurística es un campo de estudio que ofrece métodos heurísticos para resolver problemas computacionales que no tienen una solución específica. En su mayoría, estos métodos se utilizan para resolver problemas de optimización combinatoria, es decir, problemas que requieren explorar un gran espacio de posibles combinaciones para encontrar la mejor solución.

La propuesta es desarrollar una API Restful que implemente el algoritmo de búsqueda del péndulo como técnica metaheurística para resolver este problema. El algoritmo de búsqueda del péndulo es una técnica que busca la mejor solución mediante la exploración de un espacio de posibles soluciones, permitiendo obtener aproximaciones que encuentren la mejor solución en tiempos razonables y con costos computacionales aceptables.

\EOD

\end{document}
